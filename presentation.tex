\documentclass{beamer}
\usepackage{minted}
\usepackage[T1]{fontenc}
\usepackage{lmodern}
\usepackage{DejaVuSansMono}
\usepackage{upquote}
\usepackage{color}
\usepackage{xpatch,letltxmacro}
\LetLtxMacro{\cminted}{\minted}
\let\endcminted\endminted
\xpretocmd{\cminted}{\RecustomVerbatimEnvironment{Verbatim}{BVerbatim}{}}{}{}
\definecolor{light-gray}{gray}{0.90}
\begin{document}
\title{Regular Expressions Primer}
\author{Jeremy Stephens \\ Computer Systems Analyst \\ Department of Biostatistics}
\date{December 18, 2015}
\maketitle
\begin{frame}
  \begin{figure}[h]
    \centering
    \includegraphics[width=4in]{"email-regexp"}
  \end{figure}
\end{frame}
\begin{frame}
  \frametitle{What are they?}
  Regular expressions are a way to describe patterns in text.
\end{frame}
\note{The regular expression shown in the previous slide is for matching valid e-mail addresses. Hopefully by the end of the talk, what this does will be more clear.}
\begin{frame}[fragile]
  \frametitle{Why use them?}
  \begin{itemize}[<+->]
    \item To find stuff
      \vspace{3mm}

\begin{minted}[bgcolor=light-gray,fontsize=\tiny]{r}
  haystack <- c("abcdef", "anbeceddelfe", "abcdef")
  grep("n.e.e.d.l.e", haystack) #=> [1] 2
\end{minted}

      \vspace{3mm}

    \item To remove cruft
      \vspace{3mm}

\begin{minted}[bgcolor=light-gray,fontsize=\tiny]{r}
  x <- c("123", "123 oz", "123 ounces")
  sub("\\D+$", "", x) #=> [1] "123" "123" "123"
\end{minted}

      \vspace{3mm}

    \item To extract data
      \vspace{3mm}

\begin{minted}[bgcolor=light-gray,fontsize=\tiny]{r}
  x <- "The Predators lost in overtime with 4:43 left on the clock."
  gsub("^.+(\\d{1,2}:\\d{2}).+$", "\\1", x) #=> "4:43"
\end{minted}

      \vspace{3mm}

  \end{itemize}
\end{frame}
\begin{frame}[fragile]
\begin{figure}[htp]
\centering
\begin{cminted}[fontsize=\fontsize{0.12cm}{0.2cm}]{text}
(?:(?:\r\n)?[ \t])*(?:(?:(?:[^()<>@,;:\\".\[\] \000-\031]+(?:(?:(?:\r\n)?[ \t])+|\Z|(?=[\["()<>@,;:\\".\[\]]))|"(?:[^\"\r\\]|\\.|(?:(?:\r\n)?[ \t]))*"(?:(?:
\r\n)?[ \t])*)(?:\.(?:(?:\r\n)?[ \t])*(?:[^()<>@,;:\\".\[\] \000-\031]+(?:(?:(?:\r\n)?[ \t])+|\Z|(?=[\["()<>@,;:\\".\[\]]))|"(?:[^\"\r\\]|\\.|(?:(?:\r\n)?[ 
\t]))*"(?:(?:\r\n)?[ \t])*))*@(?:(?:\r\n)?[ \t])*(?:[^()<>@,;:\\".\[\] \000-\031]+(?:(?:(?:\r\n)?[ \t])+|\Z|(?=[\["()<>@,;:\\".\[\]]))|\[([^\[\]\r\\]|\\.)*\
](?:(?:\r\n)?[ \t])*)(?:\.(?:(?:\r\n)?[ \t])*(?:[^()<>@,;:\\".\[\] \000-\031]+(?:(?:(?:\r\n)?[ \t])+|\Z|(?=[\["()<>@,;:\\".\[\]]))|\[([^\[\]\r\\]|\\.)*\](?:
(?:\r\n)?[ \t])*))*|(?:[^()<>@,;:\\".\[\] \000-\031]+(?:(?:(?:\r\n)?[ \t])+|\Z|(?=[\["()<>@,;:\\".\[\]]))|"(?:[^\"\r\\]|\\.|(?:(?:\r\n)?[ \t]))*"(?:(?:\r\n)
?[ \t])*)*\<(?:(?:\r\n)?[ \t])*(?:@(?:[^()<>@,;:\\".\[\] \000-\031]+(?:(?:(?:\r\n)?[ \t])+|\Z|(?=[\["()<>@,;:\\".\[\]]))|\[([^\[\]\r\\]|\\.)*\](?:(?:\r\n)?[
 \t])*)(?:\.(?:(?:\r\n)?[ \t])*(?:[^()<>@,;:\\".\[\] \000-\031]+(?:(?:(?:\r\n)?[ \t])+|\Z|(?=[\["()<>@,;:\\".\[\]]))|\[([^\[\]\r\\]|\\.)*\](?:(?:\r\n)?[ \t]
)*))*(?:,@(?:(?:\r\n)?[ \t])*(?:[^()<>@,;:\\".\[\] \000-\031]+(?:(?:(?:\r\n)?[ \t])+|\Z|(?=[\["()<>@,;:\\".\[\]]))|\[([^\[\]\r\\]|\\.)*\](?:(?:\r\n)?[ \t])*
)(?:\.(?:(?:\r\n)?[ \t])*(?:[^()<>@,;:\\".\[\] \000-\031]+(?:(?:(?:\r\n)?[ \t])+|\Z|(?=[\["()<>@,;:\\".\[\]]))|\[([^\[\]\r\\]|\\.)*\](?:(?:\r\n)?[ \t])*))*)
*:(?:(?:\r\n)?[ \t])*)?(?:[^()<>@,;:\\".\[\] \000-\031]+(?:(?:(?:\r\n)?[ \t])+|\Z|(?=[\["()<>@,;:\\".\[\]]))|"(?:[^\"\r\\]|\\.|(?:(?:\r\n)?[ \t]))*"(?:(?:\r
\n)?[ \t])*)(?:\.(?:(?:\r\n)?[ \t])*(?:[^()<>@,;:\\".\[\] \000-\031]+(?:(?:(?:\r\n)?[ \t])+|\Z|(?=[\["()<>@,;:\\".\[\]]))|"(?:[^\"\r\\]|\\.|(?:(?:\r\n)?[ \t
]))*"(?:(?:\r\n)?[ \t])*))*@(?:(?:\r\n)?[ \t])*(?:[^()<>@,;:\\".\[\] \000-\031]+(?:(?:(?:\r\n)?[ \t])+|\Z|(?=[\["()<>@,;:\\".\[\]]))|\[([^\[\]\r\\]|\\.)*\](
?:(?:\r\n)?[ \t])*)(?:\.(?:(?:\r\n)?[ \t])*(?:[^()<>@,;:\\".\[\] \000-\031]+(?:(?:(?:\r\n)?[ \t])+|\Z|(?=[\["()<>@,;:\\".\[\]]))|\[([^\[\]\r\\]|\\.)*\](?:(?
:\r\n)?[ \t])*))*\>(?:(?:\r\n)?[ \t])*)|(?:[^()<>@,;:\\".\[\] \000-\031]+(?:(?:(?:\r\n)?[ \t])+|\Z|(?=[\["()<>@,;:\\".\[\]]))|"(?:[^\"\r\\]|\\.|(?:(?:\r\n)?
[ \t]))*"(?:(?:\r\n)?[ \t])*)*:(?:(?:\r\n)?[ \t])*(?:(?:(?:[^()<>@,;:\\".\[\] \000-\031]+(?:(?:(?:\r\n)?[ \t])+|\Z|(?=[\["()<>@,;:\\".\[\]]))|"(?:[^\"\r\\]|
\\.|(?:(?:\r\n)?[ \t]))*"(?:(?:\r\n)?[ \t])*)(?:\.(?:(?:\r\n)?[ \t])*(?:[^()<>@,;:\\".\[\] \000-\031]+(?:(?:(?:\r\n)?[ \t])+|\Z|(?=[\["()<>@,;:\\".\[\]]))|"
(?:[^\"\r\\]|\\.|(?:(?:\r\n)?[ \t]))*"(?:(?:\r\n)?[ \t])*))*@(?:(?:\r\n)?[ \t])*(?:[^()<>@,;:\\".\[\] \000-\031]+(?:(?:(?:\r\n)?[ \t])+|\Z|(?=[\["()<>@,;:\\
".\[\]]))|\[([^\[\]\r\\]|\\.)*\](?:(?:\r\n)?[ \t])*)(?:\.(?:(?:\r\n)?[ \t])*(?:[^()<>@,;:\\".\[\] \000-\031]+(?:(?:(?:\r\n)?[ \t])+|\Z|(?=[\["()<>@,;:\\".\[
\]]))|\[([^\[\]\r\\]|\\.)*\](?:(?:\r\n)?[ \t])*))*|(?:[^()<>@,;:\\".\[\] \000-\031]+(?:(?:(?:\r\n)?[ \t])+|\Z|(?=[\["()<>@,;:\\".\[\]]))|"(?:[^\"\r\\]|\\.|(
?:(?:\r\n)?[ \t]))*"(?:(?:\r\n)?[ \t])*)*\<(?:(?:\r\n)?[ \t])*(?:@(?:[^()<>@,;:\\".\[\] \000-\031]+(?:(?:(?:\r\n)?[ \t])+|\Z|(?=[\["()<>@,;:\\".\[\]]))|\[([
^\[\]\r\\]|\\.)*\](?:(?:\r\n)?[ \t])*)(?:\.(?:(?:\r\n)?[ \t])*(?:[^()<>@,;:\\".\[\] \000-\031]+(?:(?:(?:\r\n)?[ \t])+|\Z|(?=[\["()<>@,;:\\".\[\]]))|\[([^\[\
]\r\\]|\\.)*\](?:(?:\r\n)?[ \t])*))*(?:,@(?:(?:\r\n)?[ \t])*(?:[^()<>@,;:\\".\[\] \000-\031]+(?:(?:(?:\r\n)?[ \t])+|\Z|(?=[\["()<>@,;:\\".\[\]]))|\[([^\[\]\
r\\]|\\.)*\](?:(?:\r\n)?[ \t])*)(?:\.(?:(?:\r\n)?[ \t])*(?:[^()<>@,;:\\".\[\] \000-\031]+(?:(?:(?:\r\n)?[ \t])+|\Z|(?=[\["()<>@,;:\\".\[\]]))|\[([^\[\]\r\\]
|\\.)*\](?:(?:\r\n)?[ \t])*))*)*:(?:(?:\r\n)?[ \t])*)?(?:[^()<>@,;:\\".\[\] \000-\031]+(?:(?:(?:\r\n)?[ \t])+|\Z|(?=[\["()<>@,;:\\".\[\]]))|"(?:[^\"\r\\]|\\
.|(?:(?:\r\n)?[ \t]))*"(?:(?:\r\n)?[ \t])*)(?:\.(?:(?:\r\n)?[ \t])*(?:[^()<>@,;:\\".\[\] \000-\031]+(?:(?:(?:\r\n)?[ \t])+|\Z|(?=[\["()<>@,;:\\".\[\]]))|"(?
:[^\"\r\\]|\\.|(?:(?:\r\n)?[ \t]))*"(?:(?:\r\n)?[ \t])*))*@(?:(?:\r\n)?[ \t])*(?:[^()<>@,;:\\".\[\] \000-\031]+(?:(?:(?:\r\n)?[ \t])+|\Z|(?=[\["()<>@,;:\\".
\[\]]))|\[([^\[\]\r\\]|\\.)*\](?:(?:\r\n)?[ \t])*)(?:\.(?:(?:\r\n)?[ \t])*(?:[^()<>@,;:\\".\[\] \000-\031]+(?:(?:(?:\r\n)?[ \t])+|\Z|(?=[\["()<>@,;:\\".\[\]
]))|\[([^\[\]\r\\]|\\.)*\](?:(?:\r\n)?[ \t])*))*\>(?:(?:\r\n)?[ \t])*)(?:,\s*(?:(?:[^()<>@,;:\\".\[\] \000-\031]+(?:(?:(?:\r\n)?[ \t])+|\Z|(?=[\["()<>@,;:\\
".\[\]]))|"(?:[^\"\r\\]|\\.|(?:(?:\r\n)?[ \t]))*"(?:(?:\r\n)?[ \t])*)(?:\.(?:(?:\r\n)?[ \t])*(?:[^()<>@,;:\\".\[\] \000-\031]+(?:(?:(?:\r\n)?[ \t])+|\Z|(?=[
\["()<>@,;:\\".\[\]]))|"(?:[^\"\r\\]|\\.|(?:(?:\r\n)?[ \t]))*"(?:(?:\r\n)?[ \t])*))*@(?:(?:\r\n)?[ \t])*(?:[^()<>@,;:\\".\[\] \000-\031]+(?:(?:(?:\r\n)?[ \t
])+|\Z|(?=[\["()<>@,;:\\".\[\]]))|\[([^\[\]\r\\]|\\.)*\](?:(?:\r\n)?[ \t])*)(?:\.(?:(?:\r\n)?[ \t])*(?:[^()<>@,;:\\".\[\] \000-\031]+(?:(?:(?:\r\n)?[ \t])+|
\Z|(?=[\["()<>@,;:\\".\[\]]))|\[([^\[\]\r\\]|\\.)*\](?:(?:\r\n)?[ \t])*))*|(?:[^()<>@,;:\\".\[\] \000-\031]+(?:(?:(?:\r\n)?[ \t])+|\Z|(?=[\["()<>@,;:\\".\[\
]]))|"(?:[^\"\r\\]|\\.|(?:(?:\r\n)?[ \t]))*"(?:(?:\r\n)?[ \t])*)*\<(?:(?:\r\n)?[ \t])*(?:@(?:[^()<>@,;:\\".\[\] \000-\031]+(?:(?:(?:\r\n)?[ \t])+|\Z|(?=[\["
()<>@,;:\\".\[\]]))|\[([^\[\]\r\\]|\\.)*\](?:(?:\r\n)?[ \t])*)(?:\.(?:(?:\r\n)?[ \t])*(?:[^()<>@,;:\\".\[\] \000-\031]+(?:(?:(?:\r\n)?[ \t])+|\Z|(?=[\["()<>
@,;:\\".\[\]]))|\[([^\[\]\r\\]|\\.)*\](?:(?:\r\n)?[ \t])*))*(?:,@(?:(?:\r\n)?[ \t])*(?:[^()<>@,;:\\".\[\] \000-\031]+(?:(?:(?:\r\n)?[ \t])+|\Z|(?=[\["()<>@,
;:\\".\[\]]))|\[([^\[\]\r\\]|\\.)*\](?:(?:\r\n)?[ \t])*)(?:\.(?:(?:\r\n)?[ \t])*(?:[^()<>@,;:\\".\[\] \000-\031]+(?:(?:(?:\r\n)?[ \t])+|\Z|(?=[\["()<>@,;:\\
".\[\]]))|\[([^\[\]\r\\]|\\.)*\](?:(?:\r\n)?[ \t])*))*)*:(?:(?:\r\n)?[ \t])*)?(?:[^()<>@,;:\\".\[\] \000-\031]+(?:(?:(?:\r\n)?[ \t])+|\Z|(?=[\["()<>@,;:\\".
\[\]]))|"(?:[^\"\r\\]|\\.|(?:(?:\r\n)?[ \t]))*"(?:(?:\r\n)?[ \t])*)(?:\.(?:(?:\r\n)?[ \t])*(?:[^()<>@,;:\\".\[\] \000-\031]+(?:(?:(?:\r\n)?[ \t])+|\Z|(?=[\[
"()<>@,;:\\".\[\]]))|"(?:[^\"\r\\]|\\.|(?:(?:\r\n)?[ \t]))*"(?:(?:\r\n)?[ \t])*))*@(?:(?:\r\n)?[ \t])*(?:[^()<>@,;:\\".\[\] \000-\031]+(?:(?:(?:\r\n)?[ \t])
+|\Z|(?=[\["()<>@,;:\\".\[\]]))|\[([^\[\]\r\\]|\\.)*\](?:(?:\r\n)?[ \t])*)(?:\.(?:(?:\r\n)?[ \t])*(?:[^()<>@,;:\\".\[\] \000-\031]+(?:(?:(?:\r\n)?[ \t])+|\Z
|(?=[\["()<>@,;:\\".\[\]]))|\[([^\[\]\r\\]|\\.)*\](?:(?:\r\n)?[ \t])*))*\>(?:(?:\r\n)?[ \t])*))*)?;\s*)
\end{cminted}
\end{figure}
\end{frame}
\begin{frame}
  \vspace*{\fill}
  \begin{center}
  \begin{minipage}{.4\textwidth}
  {\Huge\color{red} Don't Panic!}
  \end{minipage}
  \end{center}
  \vfill % equivalent to \vspace{\fill}
\end{frame}
\begin{frame}
  \frametitle{Things you (might) need to know}
  There are 3 main regular expression standards:
  \begin{itemize}
    \item POSIX Basic Regular Expressions
    \item POSIX Extended Regular Expressions (R uses this by default)
    \item Perl-based Regular Expressions (R has support for this as well)
  \end{itemize}
\end{frame}
\begin{frame}
  \frametitle{Using regular expressions in R}
  \begin{itemize}
    \item grep('pattern', text, ...)
    \item sub('pattern', replacement, text, ...)
    \item regexpr('pattern', text, ...)
  \end{itemize}
\end{frame}
\begin{frame}[fragile]
  \frametitle{Using regular expressions in other languages}
  \begin{itemize}
    \item Ruby:
\begin{minted}[bgcolor=light-gray,fontsize=\tiny]{ruby}
  text =~ /pattern/
\end{minted}
    \item Javascript:
\begin{minted}[bgcolor=light-gray,fontsize=\tiny]{javascript}
  text.match(/pattern/)
\end{minted}
    \item Python:
\begin{minted}[bgcolor=light-gray,fontsize=\tiny]{python}
  re.match('pattern', text)
\end{minted}
  \end{itemize}
\end{frame}
\begin{frame}[fragile]
  \frametitle{The simplest regular expression}
  Plain text!
  \vspace{3mm}

\begin{minted}[bgcolor=light-gray,fontsize=\tiny]{r}
  grep("test", c("foo", "bar", "test")) #=> [1] 3
\end{minted}
  \vspace{10mm}
  The pattern "test" is a perfectly valid regular expression.
\end{frame}
\begin{frame}
  \frametitle{Metacharacters}
  A "metacharacter" in a regular expression is a character with special meaning.
\end{frame}
\begin{frame}[fragile]
  \frametitle{Your first metacharacter: dot}
  A "dot" (or period) means match any one character.
  \vspace{3mm}

\begin{minted}[bgcolor=light-gray,fontsize=\tiny]{r}
  grep("foo.bar", c("fooxbar", "foo bar", "foobar", "foo12bar")) #=> [1] 1 2
\end{minted}
  \vspace{10mm}
  In the above example, "foo.bar" is the regular expression.
\end{frame}
\begin{frame}[fragile]
  \frametitle{Your first metacharacter: dot (cont.)}
  Using multiple wildcards:
  \vspace{3mm}

\begin{minted}[bgcolor=light-gray,fontsize=\tiny]{r}
  grep("foo..bar",  c("foo12bar", "foo123bar")) #=> [1] 1
  grep("foo...bar", c("foo12bar", "foo123bar")) #=> [1] 2
\end{minted}
\end{frame}
\begin{frame}[fragile]
  \frametitle{Variable-length matching: plus}
  Use + to indicate \textbf{one or more}.
  \vspace{3mm}

\begin{minted}[bgcolor=light-gray,fontsize=\tiny]{r}
  grep("foo.+bar", c("fooxbar", "foo bar", "foobar", "foo12bar")) #=> [1] 1 2 4
\end{minted}
\end{frame}
\begin{frame}[fragile]
  \frametitle{Variable-length matching: plus (cont.)}
  Using + works on normal characters too:
  \vspace{3mm}

\begin{minted}[bgcolor=light-gray,fontsize=\tiny]{r}
  grep("a+rgh", c("argh", "aaargh", "aaaaaaaargh", "ugh")) #=> [1] 1 2 3
\end{minted}
\end{frame}
\begin{frame}
  \frametitle{Quantifiers}
  Quantifiers are metacharacters that describe "how many", like the + character.
\end{frame}
\begin{frame}[fragile]
  \frametitle{Matching zero or more: asterisk}
  Use * to indicate \textbf{zero or more}.
  \vspace{3mm}

\begin{minted}[bgcolor=light-gray,fontsize=\tiny]{r}
  grep("foo.*bar", c("fooxbar", "foo bar", "foobar", "foo12bar")) #=> [1] 1 2 3 4
\end{minted}
\end{frame}
\begin{frame}[fragile]
  \frametitle{Matching zero or one: question mark}
  Use ? to indicate \textbf{zero or one}.
  \vspace{3mm}

\begin{minted}[bgcolor=light-gray,fontsize=\tiny]{r}
  grep("abc?def", c("abdef", "abcdef", "abccdef")) #=> [1] 1 2
\end{minted}
\end{frame}
\begin{frame}[fragile]
  \frametitle{Matching \emph{n} times: curly braces}
  Use \{\} to indicate \textbf{exactly \emph{n} times}.
  \vspace{3mm}

\begin{minted}[bgcolor=light-gray,fontsize=\tiny]{r}
  grep("10{6}", c("1000", "1000000")) => [1] 2
\end{minted}
\end{frame}
\begin{frame}[fragile]
  \frametitle{Matching \emph{n} times: curly braces (cont.)}
  You can also use \{\} to indicate a range.
  \vspace{3mm}

\begin{minted}[bgcolor=light-gray,fontsize=\tiny]{r}
  grep("10{2,4}1", c("101", "1001", "10001", "100001", "1000001")) #=> [1] 2 3 4
\end{minted}
\end{frame}
\begin{frame}
  \frametitle{Intermission}
  \includegraphics[width=4in]{"sleeping-corgi"}
\end{frame}
\begin{frame}[fragile]
  \frametitle{Anchor down}
  Consider the following:
  \vspace{3mm}

\begin{minted}[bgcolor=light-gray,fontsize=\tiny]{r}
  grep("10{3}", c("100", "1000", "10000")) #=> [1] 2 3
\end{minted}
\end{frame}
\begin{frame}[fragile]
  \frametitle{Anchoring to the end: dollar sign}
  Use \$ to indicate \textbf{anchoring at the end}.
  \vspace{3mm}

\begin{minted}[bgcolor=light-gray,fontsize=\tiny]{r}
  grep("10{3}$", c("100", "1000", "10000")) #=> [1] 2
\end{minted}
\end{frame}
\begin{frame}[fragile]
  \frametitle{Anchoring to the beginning: caret}
  Use \^{} to indicate \textbf{anchoring at the beginning}.
  \vspace{3mm}

\begin{minted}[bgcolor=light-gray,fontsize=\tiny]{r}
  grep("^10{3}", c("1000", "abc1000")) #=> [1] 1
\end{minted}
\end{frame}
\begin{frame}[fragile]
  \frametitle{Using both anchors}
  You can use both \^{} and \$ to anchor at both ends.
  \vspace{3mm}

\begin{minted}[bgcolor=light-gray,fontsize=\tiny]{r}
  grep("10{3}",   c("abc1000", "1000", "10000")) #=> [1] 1 2 3
  grep("10{3}$",  c("abc1000", "1000", "10000")) #=> [1] 1 2
  grep("^10{3}$", c("abc1000", "1000", "10000")) #=> [1] 2
\end{minted}
\end{frame}
\begin{frame}[fragile]
  \frametitle{Matching groups of characters: brackets}
  Use [ ] to indicate a \textbf{group of characters}, also known as a \textbf{character class}.
  \vspace{3mm}

\begin{minted}[bgcolor=light-gray,fontsize=\tiny]{r}
  grep("ab[cdef]", c("abc", "abd", "abe", "abf", "abg")) #=> [1] 1 2 3 4
\end{minted}
\end{frame}
\begin{frame}[fragile]
  \frametitle{Matching groups of characters: brackets (cont.)}
  You can also use [ ] with a range of characters.
  \vspace{3mm}

\begin{minted}[bgcolor=light-gray,fontsize=\tiny]{r}
  grep("ab[c-f]", c("abc", "abd", "abe", "abf", "abg")) #=> [1] 1 2 3 4
\end{minted}
\end{frame}
\begin{frame}[fragile]
  \frametitle{Matching groups of characters: brackets (cont.)}
  Using [ ] with multiple ranges:
  \vspace{3mm}

\begin{minted}[bgcolor=light-gray,fontsize=\tiny]{r}
  grep("ab[c-fC-F]", c("abc", "abC", "abf", "abF", "abg")) #=> [1] 1 2 3 4
\end{minted}
\end{frame}
\begin{frame}[fragile]
  \frametitle{Matching groups of characters: brackets (cont.)}
  Mix and match ranges and characters:
  \vspace{3mm}

\begin{minted}[bgcolor=light-gray,fontsize=\tiny]{r}
  grep("ab[c-fC-F123]", c("abc", "abF", "ab1", "ab2")) #=> [1] 1 2 3 4
\end{minted}
\end{frame}
\begin{frame}[fragile]
  \frametitle{Context in character classes}
  Most metacharacters lose their meaning inside character classes:
  \vspace{3mm}

\begin{minted}[bgcolor=light-gray,fontsize=\tiny]{r}
  grep("[.+*]",  c(".", "+", "*", "x")) #=> [1] 1 2 3
  grep("[a-c-]", c("a", "b", "c", "-")) #=> [1] 1 2 3 4
\end{minted}
\end{frame}
\begin{frame}[fragile]
  \frametitle{Using quantifiers with character classes}
  It's possible to put quantifiers on character classes:
  \vspace{3mm}

\begin{minted}[bgcolor=light-gray,fontsize=\tiny]{r}
  grep("[a-z]+", c("123", "abcdef")) #=> [1] 2
\end{minted}
\end{frame}
\begin{frame}
  \begin{figure}[h]
    \centering
    \includegraphics[width=4in]{"email-regexp"}
  \end{figure}
\end{frame}
\begin{frame}[fragile]
  \frametitle{Negating a character class}
  Use \^{} inside a character class to \textbf{negate it}.
  \vspace{3mm}

\begin{minted}[bgcolor=light-gray,fontsize=\tiny]{r}
  grep("[^a-z]+", c("123", "abcdef")) #=> [1] 1
\end{minted}
\end{frame}
\begin{frame}
  \begin{figure}[h]
    \centering
    \includegraphics[width=4in]{"email-regexp"}
  \end{figure}
\end{frame}
\begin{frame}
  \frametitle{Built-in character classes}
  \begin{tabular}{|l|l|}
    \hline \textbf{Shortcut} & \textbf{Expanded} \\
    \hline \texttt{\textbackslash d} & \texttt{[0-9]} \\
    \hline \texttt{\textbackslash w} & \texttt{[a-zA-Z0-9\_]} \\
    \hline \texttt{\textbackslash s} & \texttt{[ \textbackslash t\textbackslash n\textbackslash r\textbackslash f]} \\
    \hline \texttt{\textbackslash D} & \texttt{[\textasciicircum 0-9]} \\
    \hline \texttt{\textbackslash W} & \texttt{[\textasciicircum a-zA-Z0-9\_]} \\
    \hline \texttt{\textbackslash S} & \texttt{[\textasciicircum \ \textbackslash t\textbackslash n\textbackslash r\textbackslash f]} \\
    \hline
  \end{tabular}
\end{frame}
\begin{frame}[fragile]
  \frametitle{A note about backslashes in R}
  To use built-in character classes in R, you need to \textbf{escape} the
  backslashes.
  \vspace{3mm}

\begin{minted}[bgcolor=light-gray,fontsize=\tiny]{r}
  grep("\d",  c("123", "abcdef")) #=> Error!
  grep("\\d", c("123", "abcdef")) #=> [1] 1
\end{minted}
\end{frame}
\begin{frame}[fragile]
  \frametitle{A note about backslashes in R (cont.)}
  To match a literal backslash in a regular expression in R...
  \vspace{3mm}

\begin{minted}[bgcolor=light-gray,fontsize=\tiny]{r}
  grep("\\\\",     c("123", "123\\"))   #=> [1] 2
  grep("\\\\\\\\", c("123", "123\\\\")) #=> [1] 2
\end{minted}
\end{frame}
\end{document}
